\chapter{Systematic Error Analysis}
\minitoc
\label{Cap:ErrorAnalysis}

The MINER$\nu$A experiment implements the \textit{Many-Universes Uncertainty} method to calculate the error propagation in the cross section process. This method consist of shifting the simulation $\pm\sigma$ for a specific parameter or central value. For each parameter there are two universes one for the $+\sigma$ on other for $-\sigma$ shift. The parameters are shifted based on their associated errors; for example, if the energy of a hadron candidate is estimated by Bethe-Bloch, the reconstructed energy will be shifted by $\pm30$ MeV in this two universes. For the case of the flux uncertainty is repeated 100 times for the reason that it is explained below. These universes produce their own histograms for each variable, which are used to obtain a covariance matrix. For this, an average distribution is calculated by using all the universes, including the central value (CV) universe. The covariance matrix is a square matrix obtained as follow:

\begin{equation}
    C_{ij}=\frac{1}{N}\sum^N_n((u_{i,n}-\Bar{u}_{i})-(u_{j,n}-\Bar{u}_{j})),
    \label{eq:Systematics:CovMatrix}
\end{equation}

where $N$ is the number of universes, $u_{i.n}$ is the content of the $i$ bin for the $n$ universe and $\Bar{u}_i$ is the average value of all the universes for the $i$ bin. The uncertainty for a $x$ variable for a  given $i$ bin is obtained by $\delta x_i=\sqrt{C_{ii}}$. In the case of the 2D analysis, the covariant matrix is composed by a 4-dimension matrix.

There are two types of systematic uncertainties: \textit{Lateral} and \textit{Vertical} systematics. A lateral systematic error is one in which the shift of the parameter can cause the migration of events to other bins, but the integral of the histogram does not change. For example, the reconstructed energy by Bethe-Bloch model, the energy is shifted by $\pm30$ MeV from the CV causing the migration of events to other bins. A vertical systematic generally consists of weights that produce an increase or decrease in the total number of events. These weights reduce or increase the probability that a certain type of process occurs, and the integral of the histogram changes when the shift is applied. For example, the number of non-resonant one pion production events, all the no-resonant one pion production are shifted by 5\%. 

In the following sections, a description of the systematic uncertainties and the fractional uncertainty errors for the cross sections are shown.

\section{Systematic Uncertainties}
\label{Cap:ErrorAnalysis:SystematicUnc}

In this section the systematic uncertainties are described and and grouped depending the nature of these. At the end of each sub section some examples of the group uncertainties are given.

\subsection{Cross Section Models}
\label{Cap:ErrorAnalysis:SystematicUnc:GenieIntMod}
This group includes the uncertainties about the GENIE v2.12.6 interactions models. In the GENIE models, there are parameters that has an associated error. These fall in the vertical systematic uncertainties category. Depending the type of interaction there are different knobs that can be modified. In the \textbf{Table} the the parameters with their variations are shortly described. In the \textbf{Tables} \ref{tab:ErrorAnalysis:SystematicUnc:GenieElastic}-\ref{tab:ErrorAnalysis:SystematicUnc:GenieDISmodels} the uncertainty parameters are described. 

The elastic NC interactions are described in GENIE by the Ahrens model \cite{Ahrens:PhysRevD.35.785} detailed in \ref{Cap:Simulation:GENIE}. The $M_A=1.032\pm0.036$ GeV and $\eta=0.12$ parameters from the Ahrens model are shifted. Due the high efficiency  of the cuts to remove NC interactions, these systematic uncertainties do not have any effect during the measurement. In the \textbf{Table} \ref{tab:ErrorAnalysis:SystematicUnc:GenieElastic} the parameters and the fractional uncertainty effect are shown. 

\begin{table}[!htb]
    \centering
    \begin{tabular}{c|p{2.5in}|c|c|c}
        \hline
        Parameter & Description.  & Shift (1 $\sigma$) & \multicolumn{2}{c}{Effect} \\
         & & & 1D & 2D \\
        \hline 
        EtaNCEL & Parameter that adjust $\eta$ in the elastic scattering  cross section model. & $\pm30\%$ & $\sim0\%$ & $\sim0\%$ \\ \hline
        MaNCEL & This parameter modifies the $M_A$ elastic scattering cross section model. & $\pm25\%$ & $\sim0\%$ & $\sim0\%$ \\ \hline
    \end{tabular}
    \caption{Uncertainty parameters modified for elastic interactions. Table based in \cite{GENIEUnc}.}
    \label{tab:ErrorAnalysis:SystematicUnc:GenieElastic}
\end{table}

GENIE uses the Llewellyn-Smith model to predict the CC QE interactions. For this model, the axial mass $M^{CCQE}_A$, which is the main source of uncertainty in this model. The other two knobs for the CC QE interactions comes from the selection of the vector form factors between dipole vs BBBA2005, and from nuclear effects due Pauli suppression. The variables that are more affected in the 1D analysis by these uncertainties is $W_{exp}$, it is because most part of the QE events are located in the Low $W_{exp}$ region. For the 2D analysis combination of variables that it is more affected by these uncertainties is $P^T_\mu$, $T_\pi$ and $T_\pi$, $\theta_\pi$. The \textbf{Table} \ref{tab:ErrorAnalysis:SystematicUnc:GenieCCQEmodels} shows the parameter shift, a short description and the effect in the cross section. 
 
\begin{table}[!htb]
    \centering
    \begin{tabular}{c|p{1.5in}|c|c|c}
        \hline 
        Parameter & Description.  & Shift (1 $\sigma$) & \multicolumn{2}{c}{Effect} \\
         & & & 1D & 2D \\
        \hline  
        MaCCQE & Parameter that adjust the $M_A$ in CCQE scattering, in GENIE given by the Llewellyng-Smith model. & $\pm9\%$ & $>20\%$ & $>3.7\%$ \\ \hline
        VecFFCCQEshape & Changes from BBBA to dipole. & Shape & $>3\%$ & $>0.5\%$ \\
        \hline
        CCQEPauliSupViaKF & Parameter for Pauli suppression of CCQE at low $Q^2$ &$\pm30\%$ & $>0.2\%$ & $>0.4\%$ \\ \hline
        
    \end{tabular}
    \caption{Uncertainty parameters modified for CCQE interactions. Table based in \cite{GENIEUnc}.}
    \label{tab:ErrorAnalysis:SystematicUnc:GenieCCQEmodels}
\end{table}

The resonant interactions are predicted by the Rein-Sehgal (RS) model \cite{REIN198179}. The parameters for this model are the axial $M^{RES}_A$ and vector $M^{RES}_V$ form factors. The default values for these parameters are $M^{RES}_A=0.94$ GeV and $M^{RES}_V =0.84$ GeV. The variables most affected by these two parameters in the 1D analysis are $W{exp}$ and $Q^2$, In the case of the 2D analysis $P^T_\mu$, $T_\pi$ and $T_\pi$, $\theta_\pi$ are the variables more affected. The high efficiency of the cuts to remove the Neutral Current (NC) events allows to have a null effect from the normalization of NC events from the RS model.

In GENIE, the angular distribution of the pions from $\Delta$ decay is initially modeled as anisotropic, but a bug in the MC weight calculation to transition from anisotropic to isotropic results in an overprediction of the anisotropic uncertainty. This bug should be fixed in future analyses. For more information about the angular distribution, consult \cite{Genie}. There is a parameter that makes the distribution more isotropic. The variable most affected by this variation in the 1D analysis is $\theta_\pi$ and for the 2D analysis is the combination $T_\pi$, $\theta_\pi$. \textbf{Table} \ref{tab:ErrorAnalysis:SystematicUnc:GenieRESmodels} shows the parameters with their effects on the cross-section result.

\begin{table}[!htb]
    \centering
    \begin{tabular}{c|p{1.8in}|c|c|c}
        \hline 
        Parameter & Description.  & Shift (1 $\sigma$) & \multicolumn{2}{c}{Effect} \\
         & & & 1D & 2D \\
        \hline 
        MaRES & This parameter modifies the resonance production shifting the $M_A$ parameter in the RS cross section model. & $\pm5\%$ & $>6\%$ & $>2.5\%$ \\ \hline
        MvRES & This parameter modifies the resonance production shifting the $M_V$ parameter in the RS cross section model. & $\pm3\%$ & $>4\%$ & $>1.3\%$\\ \hline
        NormNCRES & It modifies the normalization of the NC RS model. & $\pm20\%$ & $\sim 0\%$ & $\sim0\%$ \\ \hline
        Theta\_Delta2Npi & This parameter change from anisotropic to isotropic the pion angle distribution & \makecell{Anisotropy to \\ isotropy} & $>4\%$ & $>6.9\%$ \\ \hline
    \end{tabular}
    \caption{Uncertainty parameters modified for CC and NC resonant interactions. Table based in \cite{GENIEUnc}.}
    \label{tab:ErrorAnalysis:SystematicUnc:GenieRESmodels}
\end{table}

The normalization of Neutral Current (NC) and Charged Current (CC) non-resonant interactions is applied to events where 1$\pi$ or 2$\pi$ are produced. These types of interactions are modeled by the Bodek-Yang (BY) model \cite{Yang_2009}. Non-resonant interactions can occur with either a neutrino interacting with a proton or a neutron. The normalization parameters for these iterations are described in the \textbf{Table} \ref{tab:ErrorAnalysis:SystematicUnc:GenieNonRES}. The variables most affected by variations in these parameters in the 1D analysis are $W_{exp}$, $Q^2$ and $P^T_\mu$, for the 2D analysis the combinations more affected are $P^T_\mu$, $T_\pi$ and $T_\pi$, $\theta_\pi$.

\begin{table}[!htb]
    \centering
    \begin{tabular}{c|p{2.2in}|c|c|c}
        \hline 
        Parameter & Description.  & Shift (1 $\sigma$) & \multicolumn{2}{c}{Effect} \\
         & & & 1D & 2D \\
        \hline  
        Rvp1pi & Modifies the NC and CC 1$\pi$ production from non-resonant inelastic scattering for $\nu p/\Bar{\nu}n$. & $\pm4\%$ & $>2.3\%$ & $>2.3\%$\\ \hline
        Rvn1pi & Modifies the NC and CC 1$\pi$ production from non-resonant inelastic scattering for $\nu n/\Bar{\nu}p$. & $\pm4\%$ & $>2.2\%$ & $>1.2\%$\\ \hline
        Rvp2pi & Modifies the NC and CC 1$\pi$ production from non-resonant inelastic scattering for $\nu p/\Bar{\nu}n$ & $\pm50\%$ & $>4.3\%$ & $>16.3\%$\\ \hline
        Rvn2pi & Modifies the NC and CC 1$\pi$ production from non-resonant inelastic scattering for $\nu n/\Bar{\nu}p$ & $\pm50\%$ & $>11.7\%$ & $>13\%$\\ \hline
         
    \end{tabular}
    \caption{Parameter, shift value and a uncertainties description for DIS and hadronization models. Based on \cite{GENIEUnc}.}
    \label{tab:ErrorAnalysis:SystematicUnc:GenieNonRES}
\end{table}


The BY model predicts the NC and CC DIS events in GENIE for the region $Q^2 > 1\ GeV^2$ and $W<2$ GeV. The parameters $A_{TH}$, $B_{TH}$ , $C_{V_{1u}}$ and $C_{V_{2u}}$ have the most significant impact on  $W_{exp}$ and $P^T_\mu$ for the 1D analysis. For the 2D analysis the combinations more affected are $P^T_\mu$, $T_\pi$ and $T_\pi$, $\theta_\pi$. The parameter that modifies the normalization of the inclusive CC non-resonant events does not have an effect.

\begin{table}[!htb]
    \centering
    \begin{tabular}{c|p{2.1in}|c|c|c}
        \hline 
        Parameter & Description.  & Shift (1 $\sigma$) & \multicolumn{2}{c}{Effect} \\
         & & & 1D & 2D \\
        \hline  
        AhtBY & $A_{TH}$ parameter of the Bodek-Yang model. & $\pm25\%$ & $>5.5\%$ & $>0.2\%$\\ \hline
        BhtBY & $B_{TH}$ parameter of the Bodek-Yang model. & $\pm25\%$ & $>6\%$ & $>0.7\%$ \\ \hline
        CV1uBY & $C_{V_{1u}}$ parameter of the Bodek-Yang model. & $\pm30\%$ & $>2\%$ & $>0.8\%$ \\ \hline
        CV2uBY & $C_{V_{2u}}$ parameter of the Bodek-Yang model. & $\pm40\%$ & $>2\%$ & $>0.7\%$ \\ \hline
        NormDISCC & This adjust the overall of the non-resonance inclusive cross section. & $\pm20\%$ & $\sim0\%$ & $\sim0\%$ \\ \hline 
    \end{tabular}
    \caption{Parameter, description, shift value and the effect of the uncertainties for DIS models. Based on \cite{GENIEUnc}.}
    \label{tab:ErrorAnalysis:SystematicUnc:GenieDISmodels}
\end{table}

There are types of interactions or effects that are no modeled in GENIE. In the MINER$\nu$A experiment the analyzer have developed studies to studies to include interactions or nuclear effects that are not included in GENIE comparing the predictions with data. Those studies give as a result the application of a weight to the simulation increasing or decreasing the number of events, but these have an uncertainty associated. The \textbf{Tables} \ref{tab:ErrorAnalysis:SystematicUnc:CoherentandDifractive} and \ref{tab:ErrorAnalysis:SystematicUnc:MnvTune} shows the parameter, description and the effect of this uncertainties in the cross section.

\begin{table}[!htb]
    \centering
    \begin{tabular}{c|p{2in}|c|c|c}
        \hline 
        Parameter & Description.  & Shift (1 $\sigma$) & \multicolumn{2}{c}{Effect} \\
         & & & 1D & 2D \\
        \hline  
        CoherentPiUnc\_CH & Normalization to the coherent pion production in plastic scintillator & $\pm20\%$ & $>2.5\%$ & $>5..3\%$\\ \hline
        DiffractiveModelUnc & Normalization to the diffractive pion production. & $\pm50\%$ & $>1.7\%$ & $>4\%$ \\ \hline 
    \end{tabular}
    \caption{These parameters modifies the weight applied to the coherent events in scintillator and the diffractive pion production. These weights affect meanly to the $W_{exp}$ and $\theta_\mu$ variables.}
    \label{tab:ErrorAnalysis:SystematicUnc:CoherentandDifractive}
\end{table}

\begin{table}[!htb]
    \centering
    \begin{tabular}{c|p{2in}|c|c}
        \hline 
        Parameter & Description.  & \multicolumn{2}{c}{Effect} \\
         & & 1D & 2D \\
        \hline  
        Low\_Recoil\_2p2h & This parameter modifies the weight applied to simulate the effect of the 2p2h interactions & $>5.5\%$ & $>1.3\%$ \\ \hline
        RPA\_HighQ2 & RPA suppression parameter for the high $Q^2$ ($Q^2 > 0.5\ GeV^2$) region. For these region all the parameters of the particle-hole potential are shifted 1$\sigma$ and the effects are summed in quadrature. & $>6\%$ & $>1.4\%$ \\ \hline
        RPA\_LowQ2 & RPA suppression shift for the low $Q^2$ region ($Q^2 < 0.5\ GeV^2$), In this region the uncertainty is based on muon capture. & $>2\%$ & $>0.3\%$ \\ \hline 
    \end{tabular}
    \caption{These parameters modifies the weight applied to the CCQE events to model the RPA effects \cite{RPAgran2017model} and the 2p2h model \cite{2p2hRodrigues_2016}. The shift applied is not a fixed value.}
    \label{tab:ErrorAnalysis:SystematicUnc:MnvTune}
\end{table}

\textcolor{red}{Agregar algunos ejemplos o mandar a la seccion con estos errores en el apendice}


\subsection{Genie FSI Models}
\label{Cap:ErrorAnalysis:SystematicUnc:GenieFSINucleons}
In the previous section, various sources of uncertainties stemming from the neutrino interaction models were presented. Following the vertex interaction, numerous interactions can occur before the particles exit the nucleus, these are known as Final State Interactions (FSI). All these effects are simulated as best as possible. This section describes the uncertainties associated with these models. The \textbf{Table} 

\begin{table}[!htb]
    \centering
    \begin{tabular}{c|p{2in}|c|c|c}
        \hline 
        Parameter & Description.  & Shift (1 $\sigma$) & \multicolumn{2}{c}{Effect} \\
         & & & 1D & 2D \\
        \hline  
        CoherentPiUnc\_CH & Normalization to the coherent pion production in plastic scintillator & $\pm20\%$ & $>2.5\%$ & $>5..3\%$\\ \hline
        DiffractiveModelUnc & Normalization to the diffractive pion production. & $\pm50\%$ & $>1.7\%$ & $>4\%$ \\ \hline 
    \end{tabular}
    \caption{These parameters modifies the weight applied to the coherent events in scintillator and the diffractive pion production. These weights affect meanly to the $W_{exp}$ and $\theta_\mu$ variables.}
    \label{tab:ErrorAnalysis:SystematicUnc:CoherentandDifractive}
\end{table}





\textcolor{red}{Agregar algunos ejemplos o mandar a la seccion con estos errores en el apendice}
\subsection{Detector}
\label{Cap:ErrorAnalysis:SystematicUnc:Detector}
\textcolor{red}{Agregar algunos ejemplos o mandar a la seccion con estos errores en el apendice}
\subsection{Muon}
\label{Cap:ErrorAnalysis:SystematicUnc:Muon}
\textcolor{red}{Agregar algunos ejemplos o mandar a la seccion con estos errores en el apendice}
\subsection{Pion Reconstruction}
\label{Cap:ErrorAnalysis:SystematicUnc:PionReco}
\textcolor{red}{Agregar algunos ejemplos o mandar a la seccion con estos errores en el apendice}
\subsection{Flux}
\label{Cap:ErrorAnalysis:SystematicUnc:Flux}
\textcolor{red}{Agregar algunos ejemplos o mandar a la seccion con estos errores en el apendice}
\subsection{Others}
\label{Cap:ErrorAnalysis:SystematicUnc:Others}
\textcolor{red}{Agregar algunos ejemplos o mandar a la seccion con estos errores en el apendice}



\begin{itemize}
    \item \textbf{Genie Interaction Model}
    \item \textbf{Genie FSI nucleons}
    \item \textbf{Genie FSI pions}
    \item \textbf{Detector}
    \item \textbf{Cross Section Models}
    \item \textbf{}
\end{itemize}



\section{Cross Section Systematic Uncertainties}
\label{Cap:ErrorAnalysis:CrossSectionUncertainties}

