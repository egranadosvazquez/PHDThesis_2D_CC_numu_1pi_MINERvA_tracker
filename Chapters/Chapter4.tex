\chapter{Analysis}
\minitoc
\label{Cap:Analysis}
%++++++++++++++++++++++++++++++++++++++
%     Introduction 
%++++++++++++++++++++++++++++++++++++++

\section{Introduction}
\label{Cap:Analysis:Introduction}
In the \textbf{Subsection} \ref{Cap:Int:Motivation} In this thesis are presented the results for a differential cross section (1D cross section) and a double differential cross section (2D cross section). The signal definition for both is the same. For these analysis were used two techniques to do the data selection.

%++++++++++++++++++++++++++++++++++++++
%     Signal definition 
%++++++++++++++++++++++++++++++++++++++

\section{Signal Definition}
The signal definitions 
%++++++++++++++++++++++++++++++++++++++
%     Data Selection 
%++++++++++++++++++++++++++++++++++++++
\section{Data Selection}



%++++++++++++++++++++++++++++++++++++++
%     Background Sustraction
%++++++++++++++++++++++++++++++++++++++
\section{Background studies}

\subsection{Background tuning}

\subsection{Background subtraction}

%++++++++++++++++++++++++++++++++++++++
%     Migration Matrix
%++++++++++++++++++++++++++++++++++++++
\section{Migration Matrix}


%++++++++++++++++++++++++++++++++++++++
%     Ungolding
%++++++++++++++++++++++++++++++++++++++

\section{Unfolding}

\subsection{Warping Studies}

\subsection{Unfolded Distributions}

%++++++++++++++++++++++++++++++++++++++
%     Efficiency  
%++++++++++++++++++++++++++++++++++++++

\section{Efficiency}

\subsection{Efficiency correction}
