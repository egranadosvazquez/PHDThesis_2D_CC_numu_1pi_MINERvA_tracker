\chapter*{Agradecimientos}
\label{Cap:Agr1}

Quiero expresar mi más sincero agradecimiento a todas las personas y organizaciones que han contribuido al desarrollo y culminación de esta tesis.

Primeramente me gustaría agradecer a mi familia en Estados Unidos, mi mamá Ma. de Jesús, que gracias a ella y su educación soy quien soy. A mi hermana Flor por todo su apoyo durante estos mas de 4 años en estados Unidos. A mi comadre Vero y mi sobrina Daiane, que siempre estuviero ahí para apoyarme y sobre todos generar tantos bonitos y divertidos momentos. Realmente me siento muy afortunado de que estuvieran tan cerca durante mi estancia.

A mi familia y amigos en México. A mi papá, quien me enseñó a ser paciente, a trabajar y una regla que siempre me decía cuando jugaba fútbol: "Si no te juntan por no ser bueno, está bien, no todos somos buenos, pero que no sea por ser indisciplinado", la cual he implementado en todo lo que hago. A mi hermano Iván, que siempre estuvo para ayudarme con los trámites de la universidad, ir por mí al aeropuerto y tener esas divertidas pláticas cuando iba de visita a México. A mi hermana Brenda, que siempre estuvo para ayudarme cuando necesitaba consultas de salud y platicar de cualquier cosa. A mis mejores amigos Luque y Bere, que los llevo siempre a donde voy 

A María Martínez Casales, quien ha sido mi gran soporte emocional durante más de tres años. Por esas largas pláticas sobre la vida cuando salíamos a caminar en la villa y por todos los bellos momentos que hemos compartido durante estos dos años y medio. 

A mis amigos latinos en Fermilab, especialmente a Barbara, Marvin, Gonzalo e Ivan, quienes nunca dudan en ayudar a los demás y que siempre están para formar comunidad.

To the MINER$\nu$A collaborators, who always have been a very friendly community. I always received help from other students when I ask questions in the Slack channels. Special thanks to Oscar, David, Dan, Andrew, Mehreen, Amy, Alejandro and Zubair.  

Special thanks to Ben, who teaches me a lot about cross sections and coding. He has been a very important guide during my research. 

To my advisor in Fermilab, Dr. Laura Fields, for her patience and guidance during all these years. Her expertise and dedication have been instrumental in the completion of this work.

A mi asesor en México, el Dr. Julián Félix, con quien he trabajado desde hace más de diez años y quien ha sido una parte importante en mi formación como profesional. Sin duda, un ejemplo de cómo debe ser un científico y un profesor. Siempre impulsando a sus alumnos a ser mejores y a desarrollarse como científicos.
	 
\newpage