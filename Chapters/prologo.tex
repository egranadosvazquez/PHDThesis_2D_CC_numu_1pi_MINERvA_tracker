\section*{Abstract}
\label{Cap:Prlg1}
\it
The future and current neutrino oscillation experiments such as NO$\nu$A\cite{NOvA}, DUNE\cite{DUNE} and HyperK\cite{HyperK} require a precise measurement of the neutrino interactions. The neutrino oscillation experiments operate in the region of energy of a few GeV where multiple interactions produce pions in the final state. Therefore, it is crucial to have a complete understanding of the pion production by charged current neutrino interactions.


The neutrino cross section is one of the necessary parameters to measure the rate of neutrino oscillation. Having this in mind, at the beginning of 2000's the MINER$\nu$A experiment was designed with the aim to reduce the cross section uncertainties and improve the nuclear models. This experiment has the capability to measure the cross section from diverse types of interactions on 6 different materials. The detector was located on the axis of the NuMI beam where the mean neutrino energy for the Low Energy era was 3 GeV and for the Medium Energy era 6 GeV. 

The results presented in this thesis show the measurement of one differential (1D analysis) and double differential (2D analysis) cross section for charged current muon-neutrino interactions that produce one pion in the final state in the MINER$\nu$A tracker (hydrocarbon) with a neutrino beam of $<E_\nu> = 6$ GeV. 

The 1D analysis shows results for regions of pion kinetic energy never shown before in previous analyses. It implements a new pion reconstruction algorithm that allows the measurement of the kinetic energy and pion angle for pions without track in the detector. Combining the old and the new technique, the data selection efficiency increases from 6\% to 11\% with a minimal effect to the purity.

The 2D analysis shows the first double differential cross section for pion production in MINER$\nu$A. This analysis shows the combination of lepton variables with hadronic variables. The results of this analysis are compared with the tuned version of GENIE v2.12.6. The production of these results brings a better understanding of the cross section for specific kinematic regions.  



\normalfont

\newpage