\chapter{Conclusions}
\minitoc
\label{Cap:Conclusion}

\section{1D aof the kinetic energy of the pion,1DAnalysis}
The cross section is measured successfully for the variables $T_\pi$, $\theta_\pi$, $P_\mu$, $P^T_\mu$, $P^z_\mu$, $\theta_\mu$, $Q^2$ and $E_\nu$. Unfortunately, warping studies show that the variable $W_{exp}$ cannot unfold. It is due to the fact that this variable is difficult to reconstruct because of the high dependence of the $W_{exp}$ variable on $E_{had}$. 

Implementing the untracked pion algorithm increases the efficiency of data selection from 6\% to 11\%. The purity of the data selection was affected by 3\%.

The implementation of untracked pions increases the phase space of the kinetic energy of the pion, showing results never observed before in the MINER$\nu$A experiment. This result shows an overestimation of the models for this $T_\pi < $ 35 MeV region. 

The results show in general good agreement between the data and the tuned simulation MnvGENIE v4.3.1 + piston-reweight. However, disagreement is still observed in the variable $T_{/pi}$, which has been observed in previous analyzes. In the case of $W_{exp}$, it is necessary to improve the measurement of the hadronic energy to reduce the dependence of $W_{exp}$ on the models. 


\section{2D analysis}
\label{Conclusions:2DAnalysis}

The double differential cross section for charged current muon-neutrino interactions that produce a pion in the final state is successfully measured. Double differential cross sections measurements were made for the combination of variables $P^T_\mu$ vs. $P^z_\mu$, $P^T_\mu$ vs. $T_\pi$, $P_\mu$ vs. $T_\pi$, and $E_\nu$ vs. $T_\pi$. Unfortunately, the results of warping studies for $T_\pi$ vs. $\theta_\pi$ show that this combination cannot be unfolded. 

In general, for the bins with high statistics, the simulation and the data show good agreement. For the combinations where $T_\pi$ is present, it shows an over-prediction of the cross section for $T_\pi$ above 200 MeV. 

The results for the 2D analysis do not include the untracked pions. In the future, these will be implemented to increase the statistics of the data selection and reduce the statistical uncertainties for some bins, at the same time to improve the measurement of the $T_\pi$ and $\theta_\pi$ variables and obtain a confident unfolding for these variables. 

