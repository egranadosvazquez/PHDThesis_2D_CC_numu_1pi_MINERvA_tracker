\chapter{Conclusions}
\minitoc
\label{Cap:Conclusion}

\section{1D Analysis}
The cross section is measured successfully for the variables $T_\pi$, $\theta_\pi$, $P_\mu$, $P^T_\mu$, $P^z_\mu$, $\theta_\mu$, $Q^2$ and $E_\nu$. Unfortunately, warping studies show that the variable $W_{exp}$ cannot be unfolded. This is due to the fact that this variable is difficult to reconstruct because of the high dependence of the variable $W_{exp}$ on $E_{had}$. 

Implementing the untracked pion algorithm increases the efficiency of data selection from 6\% to 11\%. The purity of data selection was affected by 3\%.

The implementation of untracked pions increases the phase space of the kinetic energy of the pion, showing results never observed before in the MINER$\nu$A experiment. This result shows an overestimation of the models for this $T_\pi < $ 35 MeV region. Therefore, a weight has been implemented to allow for unfolding the data. 

The results show in general good agreement between the data and the tuned simulation MnvGENIE v4.3.1 + piston-reweight. However, disagreement is still observed in the variable $T_{\pi}$, which has been observed in previous analyzes.

The $Q^2$ cross section still shows an overestimation for the low $Q^2$ region, even when the low $Q^2$ weight suppression is carried out. 

The $\theta_\pi$ cross section shows a similar functionality to what is observed in \cite{Bercellie.131.011801}, where the simulation overestimates the cross section for $\theta_\pi$ $<$ 90$^o$. 

In general, the muon variables show good agreement between the data and the simulation. $P^T_\mu$ ans $\theta_\mu$ are hardly correlated; therefore, the shape of the cross sections is observed to agree with the simulation shapes as $theta_\mu$ and $P^T_\mu$ increase. 

In the case of $W_{exp}$, it is necessary to improve the measurement of the hadronic energy to reduce the dependence of $W_{exp}$ on the models. 

The systematic errors show that the larger contributions come from the flux, muon reconstruction and cross section models. 

\section{2D analysis}
\label{Conclusions:2DAnalysis}

The double differential cross section for charged current muon-neutrino interactions that produce a pion in the final state is successfully measured. Double differential cross sections measurements were made for the combination of variables $P^T_\mu$ vs. $P^z_\mu$, $P^T_\mu$ vs. $T_\pi$, $P_\mu$ vs. $T_\pi$, and $E_\nu$ vs. $T_\pi$. Unfortunately, the results of warping studies for $T_\pi$ vs. $\theta_\pi$ show that this combination cannot be unfolded. 

In general, for the bins with high statistics, the simulation and the data show good agreement. For the combinations where $T_\pi$ is present, it shows an overprediction of the cross section for $T_\pi$ above 200 MeV. 

The simulation results for the combination of $P^{z}_\mu$ and $P^T_\mu$ agree well with the data points. It is related to the good reconstruction of the muon variables and that a large portion of the neutrino energy is carried by the lepton after the neutrino interaction. 

The shape of the projection cross sections $T_\pi$ in the bins $E_\nu$ shows a similar shape to the results obtained in \cite{Bercellie.131.011801} and \cite{Eberly:2014mra}. 

The systematic errors show that the larger contributions come from the flux, muon reconstruction and cross section models.

The results for the 2D analysis do not include the untracked pions. In the future, these will be implemented to increase data selection statistics and reduce statistical uncertainties for some bins, while improving the measurement of variables $T_\pi$ and $\theta_\pi$ and obtaining a confident unfolding for these variables.

This is the first 2D analysis for pions produced in the MINER$\nu$A collaboration. These are the first results in the collaboration that report the 2D cross section where hadronic variables such as $T_\pi$ combined with muon variables. These results allow for a better understanding of the cross sections.

 

