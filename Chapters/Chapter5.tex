\chapter{Cross Section Results}
\minitoc
\label{Cap:xSec}

The results of the cross section measurement are presented in this chapter. The measurements report both statistical and systematic errors. The systematic errors are detailed in \textbf{Chapter} \ref{Cap:ErrorAnalysis}. Some details and observations are given in the \textbf{Sections} \ref{Cap:xSec:1DResults} and \ref{Cap:xSec:2DResults}.

\section{One Differential Cross Section Results}
\label{Cap:xSec:1DResults}

Something that must be considered for the results of $\theta_\pi$ is the different signal definition where the kinetic energy of the pion is constrained between 20 MeV and 350 MeV. The MC histograms correspond to the GENIE 2.12.6 event generator, re-weighted by the MINER$\nu$A weight MnvGENIE v4.3.1 + Pion re-weight, as explained in \textbf{Chapter} \ref{Cap:Simulation:GENIE}.

For $E_\nu$, it is observed that the prediction and the data match well in most parts of the bins. This is because most of the neutrino energy reconstruction comes from the muon momentum, which is generally well reconstructed because of the MINOS Near detector.

$T_\pi$ does not match perfectly in most parts of the bins. The first bin corresponds to a region never reported in previous analyses. The good match between the data and the MC for the first bins is also related to the effects of the pion reweight. The maximum of the distributions differ; the peak of simulation is located in the bin next to the right of the peak of data. 

In the case of $\theta_\pi$, most of the bins do not match the central value of the data and MC. The shape of the distribution is similar. The higher change of the cross section is located around 90 degrees. 

The muon variables, such as $P_\mu$, $P^t_\mu$, $P^z_\mu$ and $\theta_\mu$, agree well with the data and MC. It also explains the agreement between the data and MC for the variable $E_\nu$. 

In $Q^2$ the disagreement between the data and the MC increases in the low $Q^2$ region. This has also been reported in other analyses, such as \cite{Bercellie.131.011801} and \cite{Eberly:2014mra}. The shape is similar and the maximum distribution is observed in the same region.

The reported $W_{exp}$ distribution here is highly dependent on the reconstructed $E_{had}$. This variable is not easily reconstructed because of the smearing effect caused by neutral particles, which are not observed directly in the detector. It is also highly dependent on interactions inside the nucleus, which are not directly measured, also. All these effects are included in the systematic errors. The maximum observed between the bin edges 1200 MeV and 1300 MeV corresponds well to the $\Delta$ (1232) production. The results for this variable are not very reliable due to the findings of the warping studies, where the variable demonstrates that it cannot unfold.


\foreach \var in  {enu,mixtpi,mixthetapi_deg,pmu,ptmu,pzmu,thetamu_deg,q2,wexp}{
%\foreach \var in  {enu_true,mixtpi_true,pmu_true,ptmu_true,pzmu_true,thetamu_deg_true,q2_true,wexp_true}{
    \ifthenelse{\equal{\var}{enu}}{
      \renewcommand{\NewVar}{E_\nu}
    }{}
    \ifthenelse{\equal{\var}{mixtpi}}{
      \renewcommand{\NewVar}{T_\pi}
    }{}
    \ifthenelse{\equal{\var}{mixthetapi_deg}}{
      \renewcommand{\NewVar}{\theta_\pi}
    }{}
    \ifthenelse{\equal{\var}{pmu}}{
      \renewcommand{\NewVar}{P_\mu}
    }{}
    \ifthenelse{\equal{\var}{ptmu}}{
      \renewcommand{\NewVar}{P^T_\mu}
    }{}
    \ifthenelse{\equal{\var}{pzmu}}{
      \renewcommand{\NewVar}{P^z_\mu}
    }{}
    \ifthenelse{\equal{\var}{thetamu_deg}}{
      \renewcommand{\NewVar}{\theta_\mu}
    }{}
    \ifthenelse{\equal{\var}{q2}}{
      \renewcommand{\NewVar}{Q^2}
    }{}
    \ifthenelse{\equal{\var}{wexp}}{
      \renewcommand{\NewVar}{W_{exp}}
    }{}
    \begin{figure}
        \centering
        \includegraphics[scale=0.3]{Figures/Chapter5/CrossSection/CrossSection_\var__1Pi_BWN.png}
        \caption{$\NewVar$ 1D Cross Section .}
        \label{fig:Analysis:CrossSection:1DCrossSection\var}
    \end{figure}  
}
\pagebreak

\section{Two Differential Cross Section Results}
\label{Cap:xSec:2DResults}

As explained in the introduction of the \ref{Cap:Analysis:Introduction}, the two differential cross section results do not include the untracked pions, and the kinetic energy is restricted for the range from 35 MeV to 350 MeV. The event generator used for this analysis is also GENIE 2.12.6 re-weighted by MnvGENIE v4.3.1, this is described in the \textbf{Chapter} \ref{Cap:Simulation:GENIE}. 

The 2D distributions where $T_\pi$ is included show an overprediction in the high $T_\pi$ region. This discrepancy is also observed in the data selection and may be attributed to the poor reconstruction of kinetic energy in this region. Such reconstruction inaccuracies could be caused by inelastic interactions of the pion with the detector. 

The results for the combination of $P^T_\mu$ and $P^z_\mu$ generally show good agreement between the data and MC, even for bins with low statistics. This can be attributed to the good reconstruction of these variables. 

The case of $T_\pi$ vs $\theta_\pi$ are not reliable, similar to the case of $W_{exp}$ in the one differential cross sections, this combination of variables can not be unfolded. 

Something that should be highlighted from this study is that it is the first 2D analysis in MINER$\nu$A collaboration that shows the cross section as a function of hadronic variables, such as $T_\pi$ and $\theta_\pi$.

%\foreach \var in {loW,midW,hiW}{ 
\foreach \var in  {ptmu_vs_tpi,tpi_vs_ptmu,ptmu_vs_pzmu,pzmu_vs_ptmu,tpi_vs_pmu,pmu_vs_tpi,thetapi_deg_vs_tpi,tpi_vs_thetapi_deg,enu_vs_tpi,tpi_vs_enu}{


    \ifthenelse{\equal{\var}{pzmu_vs_ptmu}}{
      \renewcommand{\NewVar}{P^z_\mu,P^T\mu}
    }{}
    \ifthenelse{\equal{\var}{tpi_vs_thetapi_deg}}{
      \renewcommand{\NewVar}{T_\pi,\theta_\pi}
    }{}
    \ifthenelse{\equal{\var}{tpi_vs_pmu}}{
      \renewcommand{\NewVar}{T_\pi,P_\mu}
    }{}
    \ifthenelse{\equal{\var}{ptmu_vs_tpi}}{
      \renewcommand{\NewVar}{P^T_\mu, T_\pi}
    }{}
    \ifthenelse{\equal{\var}{ptmu_vs_pzmu}}{
      \renewcommand{\NewVar}{P^T\mu, P^z_\mu}
    }{}
    \ifthenelse{\equal{\var}{thetapi_deg_vs_tpi}}{
      \renewcommand{\NewVar}{\theta_\pi, T_\pi}
    }{}
    \ifthenelse{\equal{\var}{pmu_vs_tpi}}{
      \renewcommand{\NewVar}{P_\mu, T\pi}
    }{}
    \ifthenelse{\equal{\var}{tpi_vs_ptmu}}{
      \renewcommand{\NewVar}{T_\pi,P^T_\mu}
    }{}
    \ifthenelse{\equal{\var}{tpi_vs_enu}}{
      \renewcommand{\NewVar}{T_\pi,E_\nu}
    }{}
    \ifthenelse{\equal{\var}{enu_vs_tpi}}{
      \renewcommand{\NewVar}{E_\nu, T_\pi}
    }{}
    \begin{figure}
        \centering
        \includegraphics[scale=0.3]{Figures/Chapter5/CrossSection/2D_CrossSection_\var__BWN.png}
        \caption{$\NewVar$ 2D Cross Section results.}
        \label{fig:Analysis:CrossSection:2DEfficiency\var}
    \end{figure}  
}
